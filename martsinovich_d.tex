\documentclass[11pt,a4paper, serif]{moderncv}

\moderncvstyle{casual}
\moderncvcolor{blue}
%\nopagenumbers{}

\usepackage{polyglossia}
\setdefaultlanguage[indentfirst=true,forceheadingpunctuation=false]{russian}
\setotherlanguages{english}

%\usepackage[scale=0.9]{geometry}
%\setlength{\footskip}{136.00005pt}                 % depending on the amount of information in the footer, you need to

\setmainfont{PT Serif}
\setsansfont{PT Sans}
\setmonofont{PT Mono}

\name{Даниил}{Марцинович}
\title{ML/MLOps engineer}
\born{31 декабря 2001}
\phone[mobile]{+7~(965)~309~57~21}
\email{den.martsinovich@yandex.ru}
\homepage{https://MartsinovichDanya.github.io}

\social[linkedin]{daniil-martsinovich-738158146}
\social[github]{MartsinovichDanya}
\social[telegram]{DanyaMartsinovich}

\photo[64pt][0.4pt]{profile_pic_s.jpg}

\begin{document}
\makecvtitle

\section{Опыт}
\cventry{2023—н.в.}{Инженер машинного обучения}{Программный Продукт}{Москва}{}{Проекты: разработка прикладных систем на основе искусственного интеллекта для решения различных задач.\newline{}\newline{}
Инструменты: Python, Rabbit MQ, Numpy, Pandas, Matplotlib, Pytorch, MLflow, Fastapi, Whisper, Mistral, Tensorrt-LLM, llama.cpp, vLLM, OpenaAI API, PostgreSQL, FAISS, Redis, Jupyter, Telegram API.\newline{}
\begin{itemize}
\item Разработал сервис транскрибации реального времени на основе Whisper
\item Реализовал систему транскрибации с диаризацией говорящих
\item Создал чат-бот для получения стенограмм онлайн встреч
\item Работал над ускорением инференса LLM моделей, удалось уменьшить время ответа LLM в 2 раза
\item Спроектировал и разработал голосового ассистента для общения и ответов на вопросы по заранее определенному контексту (RAG)
\item Выстроил процессы разработки отдела ИИ
\item Создал RnD среду для работы DS специалистов (Jupyter Hub, Gitlab, хранилище моделей и другие компоненты)
\item Руковожу командой разработки из 3 молодых специалистов
\item Отвечаю за вывод в продакшн систем разработанных мной или моей командой
\end{itemize}}

\newpage

\section{Опыт}
\cventry{2022—2023}{Инженер по работе с данными}{Программный Продукт}{Москва}{}{Проект: глобальное обновление системы обработки больших данных.\newline{}\newline{}
Инструменты: Python, Airflow, Hadoop, Pyspark, Rabbit MQ, PostgreSQL, Clickhouse, VerticaDB, Hive, Numpy, Pandas, Jupyter.\newline{}
\begin{itemize}
\item Переписал более 50 ETL процессов c Python 2 на Python 3
\item Разработал систему уникальных скриптов для проверки полного совпадения таблиц старого и нового контура
\item Реализовал процесс сбора логов всех используемых БД в одной таблице для отслеживания операций чтения/записи
\item Выявил и устранил ряд багов и инфраструктурных проблем в существующих ETL процессах
\item Разрабатывал ETL процессы для интеграции с новыми источниками данных
\end{itemize}}

\section{Опыт}
\cventry{2021—2022}{Младший инженер по работе с данными}{Программный Продукт}{Москва}{}{Проект: развитие и поддержка распределенной системы обработки больших данных.\newline{}\newline{}
Инструменты: Python, Airflow, Hadoop, Pyspark, PostgreSQL, VerticaDB, Hive, Numpy, Pandas, Jupyter.\newline{}
\begin{itemize}
\item Поддерживал работу хранилища данных, построенного по методологии Data Vault
\item Настраивал ETL процессы для подключения новых источников данных
\item Реализовал систему автоматической проверки целостности загруженных данных
\item Разработал систему загрузки архивных данных из SQL-скриптов
\end{itemize}}

\newpage

\section{Опыт}
\cventry{2019—2021}{Бэкенд разработчик}{Фриланс}{Москва}{}{Проекты: разработка веб-приложений и чат-ботов.\newline{}\newline{}
Инструменты: Python, Flask, Django, MongoDB, SQLAlchemy, PostgreSQL, Telegram API.\newline{}
\begin{itemize}
\item Разрабатывал чат-боты для телеграм и вк
\item Реализовал уникальную систему отслеживания задач в виде телеграм-бота для юридической компании
\item Разрабатывал веб-приложения различной сложности для частных предпринимателей
\item Реализовал систему парсинга и автозаполнения документов ms office на python
\end{itemize}}

\section{Образование}
\cventry{2020—2024}{Инфокоммуникационные технологии и системы связи}{МИРЭА - Российский технологический университет}{Москва}{бакалавр}{}

\section{Языки}
\cvskillentry{Русский}{5}{родной}{}{}
\cvskillentry{Английский}{4}{C2}{}{}
\cvskillentry{Немецкий}{4}{В2}{}{}



\section{Навыки и технологии}
\begin{cvcolumns}
  \cvcolumn{}{
  	\begin{itemize}
  		\item Mistral
        \item RAG
        \item Function calling
        \item Tensorrt-LLM
        \item Nvidia NeMo
        \item vLLM
        \item llama.cpp
        \item OpenCV
        \item PIL
        \item YOLO
        \item Tesseract OCR
        \item PostgreSQL
        \item Clickhouse
        \item VerticaDB
        \item Hive
  	\end{itemize}}
  \cvcolumn{}{
  	\begin{itemize}
        \item MongoDB
        \item Redis
        \item FastAPI
        \item Flask
        \item Django
        \item Nginx
        \item Uvicorn
        \item RabbitMQ
        \item Websockets
        \item Telegram API
        \item Whisper
        \item SpeechBrain
        \item pydub
        \item Seaborn
        \item Matplotlib
  	\end{itemize}}
  \cvcolumn{}{\begin{itemize}
        \item Plotly
        \item MLFlow
        \item AirFlow
        \item Linux
        \item Docker
        \item Docker Compose
        \item Kebernetes
        \item Gitlab CI/CD
        \item Prometheus
        \item Grafana
        \item Loki
        \item Promtail
        \item Node Exporter
  	\end{itemize}}
\end{cvcolumns}
% \section{Сертификаты}
% \cvitem{Yandex}{Тренировки по ML}
% \cvitem{ФКН ВШЭ}{MLOps Bootcamp}
% \cvitem{Kaggle}{Feature Engineering}
% \cvitem{Kaggle}{Data Visualization}
\end{document}


%% end of file `template.tex'.
